% This is part of the stat-toolkit documentation
% Copyright (C) 2012,2013 Krzysztof Stachowiak
% See the file FDL for copying conditions.

\section{\texttt{aggr.h}}

	\subsection{Summary}
	The \texttt{aggr.h} library provides a set of classes for aggregating data.
	All of its contents are enclosed within the \texttt{aggr} namespace. The
	aggregation classes are derived from an abstract class \texttt{aggregator}
	which defines the concept of an aggregator. The interface is simple and
	consists of two functions: \texttt{put} and \texttt{get}.

	\begin{itemize}
		\item \texttt{put(value : double) : void}\\
			This function allows storing a value in the aggregator.
		\item \texttt{get() : double}\\
			This function returns a value that is the result of the
			underlying agregation.
	\end{itemize}

	The \texttt{get} function may be called at any time as the aggregators are
	designed to compute their results dynamically. 

	\paragraph{Note}
	A convenient typedef has been placed in the \texttt{aggr} namespace to ease
	defining types based on an abstract pointer to an aggregator:\\
	\texttt{typedef unique\_ptr<aggregator> ptr;}

	\subsubsection{Available aggregators}
	There is a set of basic classes that the library provides:

	\begin{itemize}
		\item \texttt{count} - Counts the elements that are put into the aggregator.
		\item \texttt{sum} - Sums the elements that are put into the aggregator.
		\item \texttt{min} - Stores the minimal value put into the aggregator.
		\item \texttt{max} - Stores the maximal value put into the aggregator.
		\item \texttt{mean} - Computes the mean value of the input values.
		\item \texttt{stdev} - Computes the standard deviation of the input values.
		\item \texttt{ci\_gauss} - Computes the width of the confidence
			interval defined based on the input data and a predefined
			confidence level, assuming normal distribution.
	\end{itemize}

	\subsubsection{Uniform aggregators construction}
	All the aggregators can be instantiated uniformly with use of the function:
	\texttt{create\_from\_string(str : string) : unique\_ptr<aggregator>}.
	It is used by all the command line tools that accept the aggregator definitions
	from the command line arguments. The format for the string constructing an
	aggregator is:
	\begin{center}
		\texttt{\textit{aggregator-name} [\texttt{aggregator-argument-list}]}
	\end{center}
	The function will throw a string object upon receiving an unrecognized constructor
	string.

	The aggregator names are the same as the names of the respective classes. Currently
	only one of the aggregators requires an argument, which is the \texttt{ci\_gauss}
	expecting a single argument - the confidence level.


