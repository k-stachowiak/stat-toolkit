% This is part of the stat-toolkit documentation
% Copyright (C) 2012 Krzysztof Stachowiak
% See the file FDL for copying conditions.

\documentclass{report}

\begin{document}

\title{The statistical toolkit documentation/manual}
\author{Krzysztof Stachowiak}
\date{\today}

\maketitle

\tableofcontents

\chapter{Libraries}
The libraries, which in this suite take a form of autonomous and atomic C++
header files are the foundation of the entire system. In many cases they are
uniquely tied to one of the command line tools, but others are shared between
the programs. Regardles of the particular use in the suite they are designed to
be as generic as possible and enable potiential use outside of the toolkit.
Therefore their interdependencies are minimized so that in many cases a
particular header file may be simply copied and pasted in another
project\footnote{Note however, that most of them still depends on some of the boost
libraries, so the point is not that they are lightweight, but that they may be
used separately.}.

\chapter{Command Line Interface tools}

\section{\texttt{histogram}}

\subsection{All options}
\begin{itemize}
	\item \texttt{-w} \textit{bucket-width} -- Determines the width of the
		histogram bucket. The default value is 1.0.
\end{itemize}

\subsection{Summary}
The program takes a list of numbers, one number per line and builds
a histogram of the provided distribution. The default bucket width is 1.0,
but this setting may be overriden with use of the \texttt{-w \textit{width}}
option.

The output consists of two columns. The first column is built of the bucket
center values, whereas in the second columnt the number of the results that
fell into the according bucket is given.

The program automatically generates empty buckets for the ranges, for which
there were no results, therefore the data is ready for further processing.

\subsection{Examples}
Let's assume that we have a file consisting of several numbers:
\begin{verbatim}
$ cat values1
1.0
2.0
3.0
4.0
5.0
6.0
7.0
\end{verbatim}

The default behavior of the program will be the following:
\begin{verbatim}
$cat values1 | ./histogram
1	1
2	1
3	1
4	1
5	1
6	1
7	1
\end{verbatim}

We may alter the default behavior by changing hte width of the buckets:
\begin{verbatim}
$cat values1 | ./histogram -w 3
0	1
3	3
6	3
\end{verbatim}

In order to clarify the program's behavior let's consider a case with negative values
present in the input set as well as the case of empty buckets. Let's assume another
input set:
\begin{verbatim}
$cat values2
-7.0
-2.0
-1.0
0.0
1.0
2.0
3.0
\end{verbatim}

Now let's see an output for the case the bucket is narrow enough that one of the
intervals ends up empty:
\begin{verbatim}
$cat values | ./histogram -w 2
-6	1
-4	0
-2	1
0	2
2	2
4	1
\end{verbatim}

\section{\texttt{groupby}}

\subsection{All options}
\begin{itemize}
	\item \texttt{-a} \textit{constr-string} -- defines an aggregator with a
		so called construction string.
	\item \texttt{-d} \textit{delim-char} -- defines a custom delimiter.
		The default value is the tab character.
	\item \texttt{-g} \textit{group-index} -- defines a groupping criterion.
\end{itemize}

\subsection{Summary}
The program performs SQL-like groupping aggregation of a set of data given by a
stream of tabuarized textual data.

A stream of data rows separated with a linebreak is expected. The default
field separator is the tab character, but may be altered with the
\texttt{-d \textit{delimiter}} option. The output is defined by two sets
of the data processing elments: the groupping criteria and the aggregators.
It is required that there is at least one groupping criterion and at least one
aggregator defined in the command line. Note that they will appear in the output
in the same order in which they're given in the command line.

\subsubsection{Groupping criteria}
The grouppers are equivalent to the SQL's ``group by'' statements.
Assumed that we have selected a set of grouppers for fields f1, f2, etc.,
All the input data rows that have the same values in these fields will be
considered groupped. We will be furhter saying they belong to a single group.
The fields are given by indices; in order to define a groupping criterion we use
a \texttt{-g \textit{column-index}} option.

\subsubsection{Aggregators}
The aggregators define the way in which given values for the specific fields
are to be put together. Defining an aggregator consists in providing a
\texttt{-a "\textit{field-index} \textit{aggr-constr}"} option, where the
\textit{field-index} indicates the field that is to be aggregated by the
current aggregator and \textit{aggr-constr} means the string constructing
a given aggregator. The field index is a non-negative, zero-based index
of a particular column, and the constructor string is the name of the
aggregator followed by optional, aggregator-specific arguments.

The available aggregators and their respective construction strings are
the following:
\begin{itemize}
	\item \texttt{count} -- The aggregator returning the number of values
		that have been put in it.
	\item \texttt{sum} -- Sums the values that have been put in it.
	\item \texttt{mean} -- Computes the means of the values it has been
		given.
	\item \texttt{stdev} -- The standard deviation of the population
		computer.
	\item \texttt{conf\_int\_gauss \textit{conf-level}} -- Computes
		the width of the confidence interval assuming the normal
		distribution of the values put into the aggregator.
		Requires a single argument that gives the confidence
		level.
\end{itemize}

Therefore in order to define an aggregator for the field 3 that will compute
the gaussion confidence interval at the confidence level of 0.95 the argument
line should be:

\texttt{... | ./groupby ... -a "3 conf\_int\_gauss 0.95" ...}

\subsubsection{Output format}
Let's assume that fields \texttt{f1, f2, ...} have been chosen as the
grouppers and aggregators \texttt{a1, a2, ...} have been selected.
The results will take the following form:
\begin{verbatim}
f1 f2 ... a1 a2 ...
i1 i2 ... v1 v2 ...
i3 i4 ... v3 v4 ...
\end{verbatim}

... where \texttt{i1, i2, ...} -- the "indicators" -- are the labels for
the given fields that have been captured and \texttt{v1, v2, ...} are
the computed aggregated values.

\subsection{Examples}
Let's consider a simple dataset:
\begin{verbatim}
$cat data
50	4	1.0	2.0
50	4	3.0	4.0
50	6	1.0	2.0
50	6	3.0	4.0
100	4	1.0	2.0
100	4	3.0	4.0
100	6	1.0	2.0
100	6	3.0	4.0
\end{verbatim}

Let's now take a look at different results depending on the input options.

\begin{verbatim}
$cat data | ./groupby -g1 -a "2 sum"
1	"2 sum"
4	8
6	8
\end{verbatim}

\begin{verbatim}
$cat data | ./groupby -g0 -g1 -a "2 sum" -a "3 mean"
0	1	"2 sum"	"3 mean"
50	4	4	3
50	6	4	3
100	4	4	3
100	6	4	3
\end{verbatim}

\begin{verbatim}
$cat data | ./groupby -g1 -g0 -a "2 sum" -a "3 mean"
1	0	"2 sum"	"3 mean"
4	50	4	3
6	50	4	3
4	100	4	3
6	100	4	3
\end{verbatim}

\begin{verbatim}
$cat data | ./groupby -g1 -g0 -a "3 mean" -a "2 sum"
1	0	"3 mean"	"2 sum"
4	50	3	4
6	50	3	4
4	100	3	4
6	100	3	4
\end{verbatim}

\section{\texttt{tabularize}}

\subsection{Summary}
This program performs the tabularizing of the input data as in the pivot table
tools. It is assumed that the data has been aggregated prior to the tabularizing
which may be performed with the \texttt{groupby} tool.

\texttt{tabularize} positions one-dimensional labeled series of data in a
multi-dimensional array. We assume that the input is a stream of data rows,
each of which consists of several columns. It is expected that the first row
contains the column headers as the columns are referenced by name in the results.
Note however that in the input the column indices are expected.

In order to spread the data in space we first divide the columns into the
label columns and the data columns.

\subsubsection{Label columns}
The label columns are used to determine the position, so that one or more values
from the label columns build a single \textit{coordinate}.

\subsubsection{Data columns}
The columns which aren't selected as the label columns are implicitly treated as
the data columns and the values they store is to be placed inside the output
arrays.

\subsubsection{Input}
Three dimensions are considered:

\begin{itemize}
	\item \textit{pages} -- Optionnaly multiple tables may be build. Each
		of the tables is considered a page.
	\item \textit{rows} -- The rows in the table. This set will define rows
		for each page.
	\item \textit{columns} -- The columns in the table. The columns set
		defines the columns in each of the pages.
\end{itemize}

The pages coordinate is optional, but the columns sets for the rows and columns
must be provided.

\subsubsection{How are the dimensions defined}
Let's assume that we want to define a table so that there are two row columns,
say "col1" and "col2". The way the columns of the result table will be filled
is that all the values that are placed in a given row come from the input data
rows that have the same values in the columns "col1" and "col2" respectively.

Accordingly we must define the output columns in terms of a set of input columns
so that each of the values that is placed in a given output column comes from
an input data row that has the same values in the according columns.

The same may be told about the groupping of the results into pages.

\subsubsection{How are multiple values handled}
One more thing must be said about the results. The values from all the input
columns that have become the data columns will result in separate cells in the
output table. Therefore if there are more than 1 data columns, this will result
in additional columns in the ouptut table.

\subsubsection{Output}
Let's assume that we have defined the following dimensions:
\begin{itemize}
	\item \textit{pages} -- No pages to simplify the example.
	\item \textit{rows} -- "c1" and "c2" input columns.
	\item \textit{columns} -- "c3" input column.
\end{itemize}

Let's also assume that we have 2 additional columns: "c4" and "c5", which for
the above mapping become the data columns implicitly. The resulting table will
take the following form:

\begin{verbatim}
           | c3=x, c4 | c3=x, c5 | c3=y, c4 | c3=y, c5 | ...
-----------+----------+----------+----------+----------+-----
c1=r, c2=s | v0       | v1       | v2       | v3       | ...
-----------+----------+----------+----------+----------+-----
c1=t, c2=u | v0       | v1       | v2       | v3       | ...
-----------+----------+----------+----------+----------+-----
...        | ...      | ...      | ...      | ...      | ...
\end{verbatim}

Note that for the values of the columne "c3" equal to x, which should define
a simgle column, two columns are created: one for the value from the column
"c4" and one for the value from the column "c5". The decision to expand the
columns in this way instead of expanding the rows is arbitrary.

\end{document}
