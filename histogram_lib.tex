% This is part of the stat-toolkit documentation
% Copyright (C) 2012,2013 Krzysztof Stachowiak
% See the file FDL for copying conditions.

\section{\texttt{histogram.h}}

	\subsection{Summary}
	The \texttt{histogram.h} is a simple library providing a histogram
	analysis for a stream of data. It is configured with a width of the
	interval for each of the histogram bars. It exposes a simple interface
	consisting of two functions: \texttt{put} and \texttt{get\_buckets}.

	\begin{itemize}
		\item \texttt{put(value : doule) : void}\\
			This function stores a value in the histogram.
		\item \texttt{get\_buckets() : map<double, double>}\\
			This function retrieves a map defining the current state
			of the histogram. The structure maps from the intervals'
			(buckets') centers to the counts of the values that fell
			into them.
	\end{itemize}

	\paragraph{Note}
	When the values are put into the histogram object, they are assigned 
	certain buckets, which centers are computed based on the values themselves.
	However when we retrieve the final histogram, this structure may be invalid
	due to the existence of empty buckets. Such empty buckets must be generated
	which has been judged a relatively complex task and therefore such result is
	cached. Upon each call to the \texttt{put} function the cache is invalidated
	and rebuilt when the \texttt{get\_buckets} function is called subsequently.

