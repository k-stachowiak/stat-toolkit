% This is part of the stat-toolkit documentation
% Copyright (C) 2012 Krzysztof Stachowiak
% See the file FDL for copying conditions.

\documentclass{report}

\begin{document}

\title{The statistical toolkit documentation/manual}
\author{Krzysztof Stachowiak}
\date{\today}

\maketitle

\tableofcontents

\chapter{Libraries}

\section{Overview}
The libraries, which in this suite take a form of autonomous and atomic C++
header files are the foundation of the entire system. In many cases they are
uniquely tied to one of the command line tools, but others are shared between
the programs. Regardles of the particular use in the suite they are designed to
be as generic as possible and enable potiential use outside of the toolkit.
Therefore their interdependencies are minimized so that in many cases a
particular header file may be simply copied and pasted in another
project\footnote{Note however, that most of them still depends on some of the boost
libraries, so the point is not that they are lightweight, but that they may be
used separately.}.

\section{\texttt{aggr.h}}

	\subsection{Summary}
	The \texttt{aggr.h} library provides a set of classes for aggregating data.
	All onf its contents are enclosed within the \texttt{aggr} namespace. The
	aggregation classes are derived from an abstract class \texttt{aggregator}
	which defines the concept of an aggregator. The interface is simple and
	consists of two functions: \texttt{put} and \texttt{get}.

	\begin{itemize}
		\item \texttt{put(value : double) : void}\\
			This function allows storing a value in the aggregator.
		\item \texttt{get() : double}\\
			This function returns a value that is the result of the
			underlying agregation.
	\end{itemize}

	The input values aren't stored as in case of large data sets it could drain the
	available memory and is not really necessary. Therefore the \texttt{get}
	function may be called at any time as the aggregators are designed to compute
	their results dynamically. 

	\paragraph{Note}
	A convenient typedef has been placed in the \texttt{aggr} namespace to ease
	defining types based on an abstract pointer to an aggregator:\\
	\texttt{typedef unique\_ptr<aggregator> ptr;}

	\subsubsection{Available aggregators}
	There is a set of basic classes that the library provides:

	\begin{itemize}
		\item \texttt{count} - Counts the elements that are put into the aggregator.
		\item \texttt{sum} - Sums the elements that are put into the aggregator.
		\item \texttt{mean} - Computes the mean value of the input values.
		\item \texttt{stdev} - Computes the standard deviation of the input values.
		\item \texttt{ci\_gauss} - Computes the width of the confidence
			interval defined based on the input data and a predefined
			confidence levelm, assuming normal distribution.
	\end{itemize}

	\subsubsection{Uniform aggregators construction}
	All the aggregators can be instantiated uniformly with use of the function:
	\texttt{create\_from\_string(str : string) : unique\_ptr<aggregator>}.
	It is used by all the command line tools that accept the aggregator definitions
	from the command line arguments. The format for the string constructing an
	aggregator is:
	\begin{center}
		\texttt{\textit{aggregator-name} [\texttt{aggregator-argument-list}]}
	\end{center}
	The function will throw a string object upon receiving an unrecognized constructor
	string.

	The aggregator names are the same as the names of the respective classes. Currently
	only one of the aggregators requires an argument, which is the \texttt{ci\_gauss}
	expecting a single argument - the confidence level.

\section{aggr\_array.h}

	\subsection{Summary}
	The \texttt{aggr\_array.h} library provides means of defining a multidimensional
	array of aggregators. This library is a core for the pivot table generation tool
	but may be generalized to an arbitrary number of dimensions.

	The foundation of the library is an object of the class \texttt{array}, which is
	initialized with a list of dimensions and a list of the aggregator definitions.
	The dimension definition is a list of indices defining the columns which will
	form the given dimension. By the columns forming a dimension we mean that in the
	resumting array, all the values in the given dimension will come from the input
	rows having the same values in the forming columns. The aggregator definitions
	consist of pairs of the form: column index, aggregator constructor. The
	aggregators used in this library are taken from the \texttt{aggr.h} library.
	Therefore an aggregator assigned to a given column will be collecting the values
	from the input rows from the column. This way many aggregators may draw the input
	values from the same input column.

\chapter{Command Line Interface tools}

\section{\texttt{histogram}}

	\subsection{All options}
	\begin{itemize}
		\item \texttt{-d} \textit{delimiter} -- Allows selection of a custom
			delimiter for the output data.
		\item \texttt{-w} \textit{bucket-width} -- Determines the width of the
			histogram bucket. The default value is 1.0.
	\end{itemize}

	\subsection{Summary}
	The program takes a list of numbers, one number per line and builds
	a histogram of the provided distribution. The default bucket width is 1.0,
	but this setting may be overriden with use of the \texttt{-w \textit{width}}
	option.

	The output consists of two columns. The first column is built of the bucket
	center values, whereas in the second columnt the number of the results that
	fell into the according bucket is given. By default the columns are separated
	with the tab character, but this behavior can be altered with use of the
	\texttt{-d \textit{delimiter}} option.

	The program automatically generates empty buckets for the ranges, for which
	there were no results, therefore the data is ready for further processing.

\section{\texttt{aggr}}

	\subsection{Summary}
	This trivial tool accepts a list of numbers at its standard input and performs 
	one of the basic aggregations. The aggregator is shosen by the one and only
	command line argument, which is the aggregator construction string. For the
	details on the available aggregators and their respective construction strings
	see the manual for the \texttt{aggr.h} library.

\section{\texttt{groupby}}

	\subsection{All options}
	\begin{itemize}
		\item \texttt{-a} \textit{constr-string} -- defines an aggregator with a
			so called construction string.
		\item \texttt{-d} \textit{delim-char} -- defines a custom delimiter.
			The default value is the tab character.
		\item \texttt{-g} \textit{group-index} -- defines a groupping criterion.
	\end{itemize}

	\subsection{Summary}
	The program performs SQL-like groupping aggregation of a set of data given by a
	stream of tabuarized textual data.

	A stream of data rows separated with a linebreak is expected. The default
	field separator is the tab character, but may be altered with the
	\texttt{-d \textit{delimiter}} option. The output is defined by two sets
	of the data processing elments: the groupping criteria and the aggregators.
	It is required that there is at least one groupping criterion and at least one
	aggregator defined in the command line. Note that they will appear in the output
	in the same order in which they're given in the command line.

	\subsubsection{Groupping criteria}
	The grouppers are equivalent to the SQL's ``group by'' statements.
	Assumed that we have selected a set of grouppers for fields f1, f2, etc.,
	All the input data rows that have the same values in these fields will be
	considered groupped. We will be furhter saying they belong to a single group.
	The fields are given by indices; in order to define a groupping criterion we use
	a \texttt{-g \textit{column-index}} option.

	\subsubsection{Aggregators}
	The aggregators define the way in which given values for the specific fields
	are to be put together. Defining an aggregator consists in providing a
	\texttt{-a "\textit{field-index} \textit{aggr-constr}"} option, where the
	\textit{field-index} indicates the field that is to be aggregated by the
	current aggregator and \textit{aggr-constr} means the string constructing
	a given aggregator. The field index is a non-negative, zero-based index
	of a particular column, and the constructor string is the name of the
	aggregator followed by optional, aggregator-specific arguments. For details see
	the aggregator construction in the manual for the \texttt{aggr.h} library.

	For example in order to define an aggregator for the field 3 that will compute
	the gaussion confidence interval at the confidence level of 0.95 the argument
	line should be:

	\texttt{... | ./groupby ... -a "3 ci\_gauss 0.95" ...}

	\subsubsection{Output format}
	Let's assume that fields \texttt{f1, f2, ...} have been chosen as the
	grouppers and aggregators \texttt{a1, a2, ...} have been selected.
	The results will take the following form:
	\begin{verbatim}
	f1 f2 ... a1 a2 ...
	i1 i2 ... v1 v2 ...
	i3 i4 ... v3 v4 ...
	\end{verbatim}

	... where \texttt{i1, i2, ...} -- the "indicators" -- are the labels for
	the given fields that have been captured and \texttt{v1, v2, ...} are
	the computed aggregated values.

	\subsection{Examples}
	Let's consider a simple dataset:
	\begin{verbatim}
	$cat data
	50	4	1.0	2.0
	50	4	3.0	4.0
	50	6	1.0	2.0
	50	6	3.0	4.0
	100	4	1.0	2.0
	100	4	3.0	4.0
	100	6	1.0	2.0
	100	6	3.0	4.0
	\end{verbatim}

	Let's now take a look at different results depending on the input options.

	\begin{verbatim}
	$cat data | ./groupby -g1 -a "2 sum"
	1	"2 sum"
	4	8
	6	8
	\end{verbatim}

	\begin{verbatim}
	$cat data | ./groupby -g0 -g1 -a "2 sum" -a "3 mean"
	0	1	"2 sum"	"3 mean"
	50	4	4	3
	50	6	4	3
	100	4	4	3
	100	6	4	3
	\end{verbatim}

	\begin{verbatim}
	$cat data | ./groupby -g1 -g0 -a "2 sum" -a "3 mean"
	1	0	"2 sum"	"3 mean"
	4	50	4	3
	6	50	4	3
	4	100	4	3
	6	100	4	3
	\end{verbatim}

	\begin{verbatim}
	$cat data | ./groupby -g1 -g0 -a "3 mean" -a "2 sum"
	1	0	"3 mean"	"2 sum"
	4	50	3	4
	6	50	3	4
	4	100	3	4
	6	100	3	4
	\end{verbatim}

\end{document}
