% This is part of the stat-toolkit documentation
% Copyright (C) 2012,2013 Krzysztof Stachowiak
% See the file FDL for copying conditions.

\section{\texttt{pivot}}

	\subsection{All options}
	\begin{itemize}
		\item \texttt{-a} \textit{constr-string} -- defines an aggregator with a
			so called construction string.
		\item \texttt{-d} \textit{delim-char} -- defines a custom delimiter.
			The default value is the tab character.
		\item \texttt{-D} \textit{dimension-string} -- defines one of the pivot
			table dimensions.
		\item \texttt{-n} -- Hides the dimension domain, so that instead of printing
			"col = value" only prints a value.
		\item \texttt{-h} -- Enables printing of the page, row and column
			captions in the resulting table.
		\item \texttt{-H} -- Read the first input row as the list of the input
			columns' captions.
	\end{itemize}

	\subsection{Summary}
	This program performs a pivot table analysis of the input data. It enables
	selecting 2 or 3 dimensions for the result space, selecting a set of the
	aggregators and few additional minor settings.

	\subsubsection{Dimensions}
	2 or 3 dimensions can be defined. The respective cases these are: page, row
	and column, or just row and column. Their interpretation is that all the values
	at a given coordinate, say in the given page, come from the input rows that fultil
	the dimension definition. Since the dimension is given by a pair: input column,
	value, fulfuling dimension definition means that the given input data row has
	a particular value in the column x, another particular value in the column y, etc.
	The dimension definition is done by a \texttt{-D} option followed by a string
	consisting of a white space separated list of the indices of the columns that are
	to form the given dimension. The dimensions are interpreted in the order in which
	they appear in the command line. The order is: page, row, column or just row, column
	if only two dimensions are given.

	\subsubsection{Aggregations}
	The aggregations determine, how the values in the resulting table will be obtained.
	Usually many values from the input rows will be directed to a given cell (defined
	by a set of 2 or 3 coordinates). All the values are aggregated so the result may
	be obtained. The aggregation is defined by a source column index and the aggregator
	construction string. For the details on the aggregator construction strings see
	the manual section for the \texttt{aggr} library.

	\subsubsection{Decorating the output table}
	Demending on the purpose the output table may or may not need to be described. If
	we expect further processing in the pipeline it may be more convenient to only
	print the values, whereas if the result is to be analyzed by a human, the row and
	the column captions should be added. By default the captions(headers) are not
	printed, but the behavior may be altered by adding an option: \texttt{-h} to the
	command line.

	\subsubsection{The input data header}
	The input data set may or may not contain an additional header row providing
	labels for the columns. Even though the input options (e.g. the dimension
	definitions) is always given by the column indices, the ouptut may be more
	readable if based on the original column captions (an aggregator may tor example
	be labeled "mean(result)" instead of "mean(4)". To indicate that the first input
	row should be interpreted as the header row providing the captions an option:
	\texttt{-H} must be added to the command line.

	\subsubsection{Customizing the dimension descriptor printing}
	By default the dimension caption will be of a format "column = value" for each
	of the dimension's column. This behavior may clutter the output, and therefore
	it can be disabled. In order to hide the domain of the dimension descriptor so that
	they will only consist of a "value" string for each of the dimension's columns
	append a \texttt{-n} option.

	\paragraph{Note}
	It may happen that for some cell in the pivot table (i.e. for a given page, row
	and column) there will be no results in the given input data. In such case an
	``x'' character is put in the result table.
